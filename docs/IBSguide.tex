\documentclass[]{book}
\usepackage{lmodern}
\usepackage{amssymb,amsmath}
\usepackage{ifxetex,ifluatex}
\usepackage{fixltx2e} % provides \textsubscript
\ifnum 0\ifxetex 1\fi\ifluatex 1\fi=0 % if pdftex
  \usepackage[T1]{fontenc}
  \usepackage[utf8]{inputenc}
\else % if luatex or xelatex
  \ifxetex
    \usepackage{mathspec}
  \else
    \usepackage{fontspec}
  \fi
  \defaultfontfeatures{Ligatures=TeX,Scale=MatchLowercase}
\fi
% use upquote if available, for straight quotes in verbatim environments
\IfFileExists{upquote.sty}{\usepackage{upquote}}{}
% use microtype if available
\IfFileExists{microtype.sty}{%
\usepackage{microtype}
\UseMicrotypeSet[protrusion]{basicmath} % disable protrusion for tt fonts
}{}
\usepackage[margin=1in]{geometry}
\usepackage{hyperref}
\hypersetup{unicode=true,
            pdftitle={A Reading Guide to Intuitive Biostatistics},
            pdfauthor={Nathan Brouwer},
            pdfborder={0 0 0},
            breaklinks=true}
\urlstyle{same}  % don't use monospace font for urls
\usepackage{natbib}
\bibliographystyle{apalike}
\usepackage{longtable,booktabs}
\usepackage{graphicx,grffile}
\makeatletter
\def\maxwidth{\ifdim\Gin@nat@width>\linewidth\linewidth\else\Gin@nat@width\fi}
\def\maxheight{\ifdim\Gin@nat@height>\textheight\textheight\else\Gin@nat@height\fi}
\makeatother
% Scale images if necessary, so that they will not overflow the page
% margins by default, and it is still possible to overwrite the defaults
% using explicit options in \includegraphics[width, height, ...]{}
\setkeys{Gin}{width=\maxwidth,height=\maxheight,keepaspectratio}
\IfFileExists{parskip.sty}{%
\usepackage{parskip}
}{% else
\setlength{\parindent}{0pt}
\setlength{\parskip}{6pt plus 2pt minus 1pt}
}
\setlength{\emergencystretch}{3em}  % prevent overfull lines
\providecommand{\tightlist}{%
  \setlength{\itemsep}{0pt}\setlength{\parskip}{0pt}}
\setcounter{secnumdepth}{5}
% Redefines (sub)paragraphs to behave more like sections
\ifx\paragraph\undefined\else
\let\oldparagraph\paragraph
\renewcommand{\paragraph}[1]{\oldparagraph{#1}\mbox{}}
\fi
\ifx\subparagraph\undefined\else
\let\oldsubparagraph\subparagraph
\renewcommand{\subparagraph}[1]{\oldsubparagraph{#1}\mbox{}}
\fi

%%% Use protect on footnotes to avoid problems with footnotes in titles
\let\rmarkdownfootnote\footnote%
\def\footnote{\protect\rmarkdownfootnote}

%%% Change title format to be more compact
\usepackage{titling}

% Create subtitle command for use in maketitle
\newcommand{\subtitle}[1]{
  \posttitle{
    \begin{center}\large#1\end{center}
    }
}

\setlength{\droptitle}{-2em}

  \title{A Reading Guide to Intuitive Biostatistics}
    \pretitle{\vspace{\droptitle}\centering\huge}
  \posttitle{\par}
    \author{Nathan Brouwer}
    \preauthor{\centering\large\emph}
  \postauthor{\par}
      \predate{\centering\large\emph}
  \postdate{\par}
    \date{2018-08-13}

\usepackage{booktabs}
\usepackage{amsthm}
\makeatletter
\def\thm@space@setup{%
  \thm@preskip=8pt plus 2pt minus 4pt
  \thm@postskip=\thm@preskip
}
\makeatother

\usepackage{amsthm}
\newtheorem{theorem}{Theorem}[chapter]
\newtheorem{lemma}{Lemma}[chapter]
\theoremstyle{definition}
\newtheorem{definition}{Definition}[chapter]
\newtheorem{corollary}{Corollary}[chapter]
\newtheorem{proposition}{Proposition}[chapter]
\theoremstyle{definition}
\newtheorem{example}{Example}[chapter]
\theoremstyle{definition}
\newtheorem{exercise}{Exercise}[chapter]
\theoremstyle{remark}
\newtheorem*{remark}{Remark}
\newtheorem*{solution}{Solution}
\begin{document}
\maketitle

{
\setcounter{tocdepth}{1}
\tableofcontents
}
\chapter*{Preface}\label{preface}
\addcontentsline{toc}{chapter}{Preface}

This is a reading guide to Harvey Motulsky's
\href{http://www.intuitivebiostatistics.com/}{Intuitive Biostatistics: A
Nonmathematical Guide to Statistical Thinking}, 4th edition. More
information about the book can be found at the
\href{http://www.intuitivebiostatistics.com/}{book's website},
\url{http://www.intuitivebiostatistics.com/}, and it can be purchased
from
\href{https://www.amazon.com/Intuitive-Biostatistics-Nonmathematical-Statistical-Thinking/dp/0190643560/ref=asap_bc?ie=UTF8}{Amazon.com}.
Motulsky is the CEO and Founder of
\href{https://www.graphpad.com/}{GraphPad}, a user-friendly statistical
software popular in some branches of the life sciences.

\emph{Intutitive Biostatistics} is a fabulous book for researchers that
need to understand or do basic statistics and either need a concise
primer on the key issues and/or are turned off by the equations
underlying the statistical methods. Instead of using math to explain
statistical methods, Motulsky focuses on written explanations,
real-world examples, and novel graphing approaches. An excellent aspect
of this book is that it unpacks common misunderstandings that
researchers have, such as how to interpret p-values (Chapter 17), and
signposts bad practices that must be avoided (like p-hacking). Again,
this is done by focusing on intuition, not math. Motulsky also presents
best practices in plotting, data presentation, and data reporting,
emphasizing the key aspect of adequate and accurate presentation of
results.

This reading guide serves several purposes:

\begin{itemize}
\tightlist
\item
  Highlight the parts of the book I focus on in my teaching (and so will
  be on any tests!)
\item
  Provide additional complementary examples
\item
  Indicate extensions or alternatives
\item
  Provide citations and links to resources for follow-up
\item
  Indicate where others (including myself, though I am not a trained
  statistician) might disagree with Motulsky
\end{itemize}

Each part of the reading guide is essentially an outline of each chapter
with commentary as needed. In some cases I have written a brief initial
commentary to put the chapter in context. I will often indicate the
Excel or R functions related to methods or calculations; for a fuller
treatment see my other guide \emph{An R Companion to Motulsky's}
Intuitive Biostatistics. At the end of each chapter are typically
references, a list of R and Excel functions needed to carry out the
analyses in the book, and study questions to consider.

My most important notes and comments are generally in \textbf{bold} or
bulleted. When I've riffed on an idea and its not necessarily key I've
usually put in in a block quote, like the one below:

\begin{quote}
For example, sometimes I've written about a section, and my text is
almost as long as the original section!
\end{quote}

This is a work in progress and many sections are not yet annotated; feel
free to contact me with suggestions or corrections.

Nathan Brouwer
\href{mailto:brouwern@gmail.com}{\nolinkurl{brouwern@gmail.com}}

\chapter{\texorpdfstring{``Statistics \& Probablity Are Not
Intuitive''}{Statistics \& Probablity Are Not Intuitive}}\label{intro1}

\section*{Commentary}\label{commentary}
\addcontentsline{toc}{section}{Commentary}

In this introductory chapter Motulsky sketches out some major reasons
why people struggle with statistics and probability. This chapter
assumpes some basic familiarity with statistical ideas. Sometimes this
chapter is a bit terse - its meant to highlight key ideas, not fully
discuss or demonstate them.

\section*{Vocabulary}\label{vocabulary}
\addcontentsline{toc}{section}{Vocabulary}

\subsection*{Motulsky vocab}\label{motulsky-vocab}
\addcontentsline{toc}{subsection}{Motulsky vocab}

\begin{itemize}
\tightlist
\item
  sample
\item
  population
\item
  Bayesian
\item
  multiple comparisons
\item
  regression to the mean
\end{itemize}

\subsection*{Additional vocab}\label{additional-vocab}
\addcontentsline{toc}{subsection}{Additional vocab}

\begin{itemize}
\tightlist
\item
  Bayes theorem
\item
  pre-registration
\item
  exploratory analyses
\end{itemize}

\subsection*{Key functions}\label{key-functions}
\addcontentsline{toc}{subsection}{Key functions}

None

\section*{Chapter Notes}\label{chapter-notes}
\addcontentsline{toc}{section}{Chapter Notes}

\section{We Tend to Jump to
Conclusions}\label{we-tend-to-jump-to-conclusions}

\begin{quote}
Motulsky uses the phrase ``\textbf{generalize from a sample to a
population}'' without defining what this means. In general, this means
to look at some subset of the world - either something experienced in
real life or generated using a scientific study - and conclude that what
was seen in the subset occurs elsewhere. In the example he uses, his
daughter experienced meeting doctors, and they all were male, so she
generalized to the rest of the world that all doctors must be male.
While this example is trivial, anytime we generalize from sample to
population (or from a part to the whole) we run the risk that our sample
is biased. It could be biased becauase we didn't take a good sample,
such as relying just on personal experiencce. Or it could be a
rigorously collected scientific sample, but still be non-representative.
What if he wanted to prove his daughter wrong and so randomly selected
10 doctor's offices for a web search and looked up who the senior
physician is. If he happend to find my doctor's office, he'd see that
its a women, Dr.~Cathy Lamb. However, it is possible that he could look
up 10 doctor's and they could all still be male.
\end{quote}

\section{We Tend to Be Overconfident}\label{we-tend-to-be-overconfident}

\section{We see Patterns in Random
Data}\label{we-see-patterns-in-random-data}

\section{We don't realize that coincidences are
common}\label{we-dont-realize-that-coincidences-are-common}

He doesn't use the specific term, but he is alluding to the concept of
\textbf{hindsight bias}.

\section{We don't expect variability to depend on sample
size}\label{we-dont-expect-variability-to-depend-on-sample-size}

Motulsky cites a paper by Andrew Gelman here, one of the most thought
provoking - though sometimes just provoking -- statistics bloggers of
the last decade. He blogs regularly at
\href{http://andrewgelman.com/}{Statistical Modeling, Causal Inference,
and Social Science} and writes non-technical pieces for a number of
outlets, including
\href{http://www.slate.com/authors.andrew_gelman.html}{Slate}. He is
also prominent Bayesian.

\section{We Have Incorrect Intuitive Feelings About
Probability}\label{we-have-incorrect-intuitive-feelings-about-probability}

\section{We Find it Hard to Combine
Probabilities}\label{we-find-it-hard-to-combine-probabilities}

\section{(We Avoid Thinking About Ambiguous
Situations)}\label{we-avoid-thinking-about-ambiguous-situations}

(This section appear in previous versions; I am not sure where/if it
occurs in the 4th edition)

\section{We Don't Do Bayesian Calculations
Intuitively}\label{we-dont-do-bayesian-calculations-intuitively}

\begin{quote}
Motulsky doesn't define \textbf{Bayesian} here, though its not central
to what he's talking about. In this example, ``Bayesian calculations''
refers to a particular type of probability calculation using
\textbf{Bayes Rule}. His example is a classic example of how probability
calculations are used for diagnostic testing.
\end{quote}

\begin{quote}
More generally, ``Bayesian''" refers to a particular way to use the
mathematics of probability to make inference. All mathematicians agree
on the basic rules of probability calculations. In contrast, when it
comes to using the math of probablity to make inference from a sample to
a population - that is, to do statistics - there is a huge rift between
\textbf{Frequentists} and \textbf{Bayesians}.
\end{quote}

\section{We are Fooled By Multiple
Comparisons}\label{we-are-fooled-by-multiple-comparisons}

The study on astroglogical signs here is a great paper intended to ``To
illustrate how multiple hypotheses testing can produce associations with
no clinical plausibility'' (Austin et al 2006, Abstract). ``Multiple
hypotheses testing'' means the same thing as ``multiple comparisons.''
As Motulsky indicates, if you test multiple hypotheses or make multiple
comparisons between things, sooner or later you'll find a strong
association. This is why it important to make specific hypotheses prior
to the beginning of a study - ideally even publically
\textbf{pre-registering} them - and properly indicate which analyses
were defined in advance and which are \textbf{exploratory analyses}.

\textbf{Multiple Comparisons} is a big topic that Motulsky doesn't go
into detail yet. He devotes several excellent chapters to this topic
elsewhere. This issue of multiple comparisons is a big and controversial
one. For a discussion of multiple comparisons

\section{We tend to ignore alternative
explanations}\label{we-tend-to-ignore-alternative-explanations}

\section{We are fooled by regression to the
mean}\label{we-are-fooled-by-regression-to-the-mean}

\textbf{Regression to the mean} is a concept that isn't typically taught
in intro stats courses, especially for ecology. For its relevance to
ecology and evolution see the paper by Kelly and Price (2006)
\href{https://www.journals.uchicago.edu/doi/abs/10.1086/497402}{``Correcting
for Regression to the Mean in Behavior and Ecology''} in \emph{American
Naturalist}.

\section{We let our biases determine how we interpret
data}\label{we-let-our-biases-determine-how-we-interpret-data}

\section{We crave certainty, but statistics offers
probability}\label{we-crave-certainty-but-statistics-offers-probability}

\section{Further reading}\label{further-reading}

\section{References}\label{references}

Austin, Mamdani, Juurlink and Hux 2006. Testing multiple statistical
hypotheses resulted in spurious associations: a study of astrological
signs and health. Journal of Clinical Epidemiology 59:964--969
\href{https://www.jclinepi.com/article/S0895-4356(06)00124-7/abstract?code=jce-site}{Open
Access}

\section{Annotated Bibliography}\label{annotated-bibliography}

\subsection{Multiple comparisons}\label{multiple-comparisons}

Bender \& Lange 2001. Adjusting for multiple testing---when and how?
Journal of Clinical Epidemiology. 54:343--349.
\href{https://www.jclinepi.com/article/S0895-4356(00)00314-0/abstract?code=jce-site}{Abstract}

\begin{quote}
\textbf{Multiple comparisons} is a thorny issue that Motulsky briefly
introduces here in Chapter 1 and discusses in depth elsewhere.
Throughout the book Motulsky focuses on the need for multiple
comparisons procedures in general, and the most popular ones used; he
doesn't go into the broader arguements about their use and the many ways
they can be problematic. Bender \& Lange (2001) give a taste of the mess
made by multiple comparisons issues. They note ``\ldots{}there seems to
be a lack of knowledge about statistical procedures for multiple
testing. For instance, multiple test adjustments have been equated with
the Bonferroni procedure, which is the simplest, but frequently also an
inefficient method \ldots{}'' (pg. 343). They discuss the various
positions that have been taken for and against multiple comparisons in
the biomedical sciences, and advance their particular perspective on the
issue. Elsewhere in the book Motulsky discusses the Bonferonni
correction under the heading ``The Traditional Approach to Correcting
For Multiple Comparisons.'' He then outlines a more contemporary
approach, the \textbf{False Discovery Rate (FDR)}. Bender \& Lange
(2001) was written before the FDR became popular and instead briefly
disucuss other alternatives, including Holm modificaiton to the
Bonferroni procedures and advanced computational methods.
\end{quote}

\chapter{\texorpdfstring{Chapter 2: ``The complexities of
probability''}{Chapter 2: The complexities of probability}}\label{intro2}

\section*{Commentary}\label{commentary-1}
\addcontentsline{toc}{section}{Commentary}

Probability is central to statistics, but its inherently hard. Most
introductory stats books spend at least one chapter to lay out the
foundations, which can seem tangential to the main task at hand -
analzying data! Advanced stats books typically go back to probability,
often in calculations that are unfortunatley not within the comfort zone
of most biologists. Motulsky doesn't shirk the responsiblity of
reviewing probability, but does so in a conversational style.

\section*{Vocabulary}\label{vocabulary-1}
\addcontentsline{toc}{section}{Vocabulary}

\subsection*{Motulsky vocab}\label{motulsky-vocab-1}
\addcontentsline{toc}{subsection}{Motulsky vocab}

\begin{itemize}
\tightlist
\item
  probablity as long-term frequency
\item
  probability as subjective belief
\item
  model
\end{itemize}

\subsection*{Aditional vocab}\label{aditional-vocab}
\addcontentsline{toc}{subsection}{Aditional vocab}

\subsection*{Key functions}\label{key-functions-1}
\addcontentsline{toc}{subsection}{Key functions}

None

\section*{Chapter Notes}\label{chapter-notes-1}
\addcontentsline{toc}{section}{Chapter Notes}

\section{Basics of probability}\label{basics-of-probability}

\section{Probability as long-term
frequency}\label{probability-as-long-term-frequency}

\subsection{Probabilities as predictions from a
model}\label{probabilities-as-predictions-from-a-model}

\textbf{model}

\subsection{Probabilities based on
data}\label{probabilities-based-on-data}

\section{Probabilities As Strength of
Belief}\label{probabilities-as-strength-of-belief}

\subsection{Subjective probabilities}\label{subjective-probabilities}

\subsection{\texorpdfstring{``Probabilities'' used to quantify
ignorance}{Probabilities used to quantify ignorance}}\label{probabilities-used-to-quantify-ignorance}

This is very important point.

\subsection{Quantitative predictions of one-time
events}\label{quantitative-predictions-of-one-time-events}

At times, when there is a one-time event someone will say something
like: ``the probability is 50\%: it either will happen or not.'' This is
a confusion of the fact that the outcomes are binary (yes/no) with the
probability that one outcome will happen or not.

The polling around the 2016 elections has provided lots of fodder for
commentary on statistics and data analysis. Andrew Gelman has blogged on
this on his own site and also for Slate. See {[}``We Need to Move Beyond
Election-Focused Polling''{]}
(\url{http://www.slate.com/articles/technology/future_tense/2017/09/what_is_the_future_of_polling.html})
which has the tagline ``Polling didn't fail us in 2016, but what
happened made polling's flaws more apparent. Here's how to fix that.''

Also see
\href{http://www.slate.com/articles/news_and_politics/politics/2016/12/_19_lessons_for_political_scientists_from_the_2016_election.html}{``19
Lessons for Political Scientists From the 2016 Election''}.

Among political-science orientated statisticians like Gelman the work of
\href{}{FiveThirtyEight.com} comes up a lot. I'm not that familiar with
it so I checked
\href{https://en.wikipedia.org/wiki/FiveThirtyEight}{Wikipedia}:
``FiveThirtyEight\ldots{}is a website that focuses on opinion poll
analysis, politics, economics, and sports blogging. The
website\ldots{}takes its name from the number of electors in the United
States electoral college.''

\section{Calculations with probabilities can be easier if you switch to
calcualting with whole
numbers}\label{calculations-with-probabilities-can-be-easier-if-you-switch-to-calcualting-with-whole-numbers}

\section{Common Mistakes:
Probability}\label{common-mistakes-probability}

\subsection{Mistake: Ignoring
assumptions}\label{mistake-ignoring-assumptions}

\subsection{Mistake: Trying to understand probability without clearly
defining both the numerator \& the
denominator}\label{mistake-trying-to-understand-probability-without-clearly-defining-both-the-numerator-the-denominator}

\subsection{Mistake: Reversing probability
statements}\label{mistake-reversing-probability-statements}

\subsection{Mistake: Believing the probability has a
memory}\label{mistake-believing-the-probability-has-a-memory}

\textbf{gambler's fallacy}

\section{Lingo}\label{lingo}

\subsection{Probability vs.~odds}\label{probability-vs.odds}

\subsection{Probability vs.~statistics}\label{probability-vs.statistics}

This is a key idea that I don't think I've though about a lot: *
probablity: general principals -\textgreater{} specific situation *
statistics: general population \textless{}- specific dataset

To relate to his earlier example, if we are interested in the
probability of a child being born XY, you can start with a general model
(how meiosis works) or data on large population (the CIA database) and
make an inference about a specific situation: the birth of a particular
child.

\subsection{Probability vs.~likelihood}\label{probability-vs.likelihood}

As Motulsky mentions, \textbf{likelihood} has a particular technical
meaning in statistics. While this his book doesn't devel into it, you
don't have to spend much time doing analyses these days before
encountering it. The following topics all involve likelihoods in their
current application:

\begin{itemize}
\tightlist
\item
  logistic regression
\item
  analysis of count data with Poisson regression
\item
  generalized linear models (GLMs; of which logistic and Poisson
  reression are forms)
\item
  mixed models
\item
  generalized linear mixed models (GLMMs)
\item
  Phylogenetic methods (estimating phylogenetic trees; using phylogeneis
  in statistica analyses)
\item
  Bayesian methods
\end{itemize}

\section{Probability In Statistics}\label{probability-in-statistics}

\textbf{Table 2.1} is a good summary. A great question on a test would
be to blank out some of the words and ask students to fill them in.

\section{Further reading}\label{further-reading-1}

\subsection{References}\label{references-1}

\subsection{Annotated Bibliography}\label{annotated-bibliography-1}

\subsubsection{Multiple comparisons}\label{multiple-comparisons-1}

\bibliography{book.bib}


\end{document}
